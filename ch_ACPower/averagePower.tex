\begin{itemize}
\item Helpful to introduce some additional ways of talking about amplitude
  \begin{itemize}
  \item The amplitude we've been using is the {\bf peak} amplitude.  As an example, the peak voltage would be written as $V_p$.
  \item Another useful amplitude is the {\bf peak-to-peak} amplitude.  This measures the total distance between peaks, and is typically written as $V_{pp}$.  The peak-to-peak voltage is twice the peak amplitude.
  \item Finally, we have the {\bf RMS} amplitude.  RMS stands for root-mean-square, which refers to the method used to obtain it (find more information in the Appendix). For sinusoidal voltage, $V_{RMS}=\frac{1}{\sqrt{2}}V_p$.  For non-sinusoids, the RMS voltage will be different.
    %$V_{RMS} = \sqrt{\frac{1}{T} \int_0^T v(t)^2 dt}$.  
  \end{itemize}
\item The RMS voltage and current are directly used in finding the power.  Watt's law for AC gives that the average power is the product of the root-mean-square voltage and current multiplied by the cosine of the phase offset between the two: $P_{avg}=V_{RMS}I_{RMS}\cos(\phi_V-\phi_I)$.  Translated to peak amplitudes (which we normally use), $P_{avg}=\frac{1}{2}V_pI_p\cos(\phi_V-\phi_I)$
\end{itemize}
