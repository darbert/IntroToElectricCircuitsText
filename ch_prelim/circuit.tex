A circuit is like a graph in that it contains nodes and branches.  For a circuit, a node is any set of locations that is connected by an {\bf ideal wire}.
Charges can pass through ideal wires without any resistance, which means that they aren't losing any energy.  Another way of phrasing it is that the voltage at every point in the wire will be the same: every charge is going to effectively have the same amount of energy in the node.

Branches are any circuit element that is not an ideal wire.  We'll talk later in detail about resistors, voltage sources, and current sources, which are the most common elements used in this class.  Charges that move through these sources will lose or gain energy, and therefore change their voltage.

The final detail about a circuit is that they usually contain a complete loop.  We very rarely have charges accumulate in specific nodes, which means that for charges to move around, they need to move around in a complete circle.  We occasionally have {\bf open circuits}, which do not make a complete loop.  These circuits do not allow any movement of charge, and therefore have no current at all.
