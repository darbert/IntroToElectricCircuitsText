
\subsection*{What is electric charge?}
\begin{itemize}
\item Fundamental property of matter
\item Most matter is neutral on a macroscopic level
\item Atoms are composed of:
  \begin{itemize}
  \item protons, which have a positive charge
  \item electrons, which have a negative charge
  \item neutrons, which do not possess a net charge
  \end{itemize}
\item The charge of an electron is constant and the most fundamental unit of charge.  However, it is very small
\item A more useful unit of charge is the Coulomb, which is equivalent to roughly $6.24\times 10^{18}$ electrons.
\end{itemize}

\subsection*{Current, Voltage, Resistance}
The three concepts we will be most interested in throughout this book are current, voltage, and resistance.

Current is:
\begin{itemize}
\item The amount of charge moving through a region (usually a wire) in a given amount of time.
\item Measured in Amperes -- Coulombs per second (1 A = 1 C/s)
\end{itemize}

Voltage is:
\begin{itemize}
\item The amount of energy present in a given amount of charge
\item Also called ``electric potential''
\item Think about a ball on a hill, which has some potential energy.  In this analogy, the mass of the ball is like the charge of our particle, the height of the hill is the voltage, and the potential energy is the electric energy of the particle.
\item Measured in Volts -- Joules per Coulomb (1 V = 1 J/C)
\end{itemize}

Resistance is:
\begin{itemize}
\item A measure of the amount of energy it takes to get some charge through a circuit element in a certain amount of time
\item Measured in Ohms, which does not have a useful conversion to more basic units
\item Think of a very thin pipe. In order to get water through the pipe, you will need to push harder and harder as the pipe gets thinner (or, alternatively, you will have to push harder in order to get more water through the same pipe).  The pipe has some fluid resistance, just as our circuit elements could have some electric resistance.
\item We will cover this more when we get to Ohm's Law
\end{itemize}
